\documentclass{report}

\usepackage{cancel}
\usepackage{amsmath}
\usepackage{ragged2e}
\usepackage[utf8]{inputenc}
\usepackage[english,russian]{babel}
\usepackage[a4paper, left=0.4cm, top=0.5cm, right=0.4cm]{geometry}





\begin{document}


\noindent
\textbf{Векторная функция} векторного аргумента -- это соответствие $r$, при котором
$\forall$ точке $x \in \Omega$ евклидова пространства $R^{m}$ сопостовляется 
вектор $r(x)$ множества $Q$ евклидова пространства $R^{p}$.\\
$x \in \Omega = \{(x_1, \ldots, x_m)\} \subset R^{m} \to r(x)
\in Q = \{(r_1, \ldots, r_p)\} \subset R^{p}$\\
При этом множество $\Omega$ область задания, а $Q$ множество значений.
Если $\Omega = \{x\}$ -- множество точек на прямой, то имеем функцию одного
скалярного аргумента $r(x)$.\\
Если $\Omega = \{(x_1, \ldots, x_m)\} \subset R^{m}$ -- множество точек евклидова
пространства, то имеем векторную функцию нескольких скалярных
аргументов $r(x_1, \ldots, x_m)$.\\

\noindent
\textbf{Годограф векторной функции}\\
Пусть $(r_1, \ldots, r_p)$ -- координаты $r(x) \in Q \subset R^{p}$.
Задание векторной функции $r(x)$ равносильно заданию скалярных функций\\
$r_1(x_1, \ldots, x_m), \ldots, r_p(x_1, \ldots, x_m)$, и если начала этих
векторов совместить с началом соответствующей ДПСК, то точечное множество концов
рассматриваемых радиус векторов будем называть годографом векторной функции.\\
If p = 3 годограф векторной функции есть кривая, p = 2 -- поверхность.\\


\noindent
\textbf{Способы задания кривых}\\
\indent Элементарной кривой называют множество точек пространства, являющееся
образом отрезка при топологическом отображении его в пространство.\\
Точки соответствующие конечным точкам отрезка, называют конечными точками
элементарной кривой. Элементарные кривые -- примыкающие если одна или обе
пары их конечных точек совпадают между собой.\\
Кривой линией называется множество точек пространства, которое состоит
из конечного или счетного множества элементарных кривых, примыкающих друг к другу.\\
\indent Пусть $\gamma$ -- элементарная кривая, являющаяся образом промежутка 
$a < t < b$ при топологическом отображении f его в пространство $R^{3}$.
$x(t),\ y(t),\ z(t)$ -- координаты точки на кривой $\gamma$ соответствующей
значению $t \in (a, b)$.\\
Тогда систему равенств $x(t),\ y(t),\ z(t),\ t \in (a, b)$ называют уравнениями кривой 
$\gamma$ в параметрической форме или параметризацией кривой (кривая $\gamma$ параметризована
этими уравнениями).\\
Если же считать $x(t),\ y(t),\ z(t)$ координатами радиус-вектора $\overrightarrow{r}(t)$
соответствующей точки кривой $\gamma$, мы получим векторную функцию $\overrightarrow{r}(t),\ $
$t \in (a, b)$, годографом которой является данная кривая. (способ задания кривой через векторную
функцию скалярного аргумента по сути эквивалентный параметрическому способу).\\


\end{document}
